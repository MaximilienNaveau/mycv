% filepath: /home/mnaveau/devel/workspace_gepetto_nix/mycv/src/cv_french.tex
%% Début du fichier cv_french.tex

\documentclass[11pt,a4paper]{moderncv}
\moderncvtheme[blue]{classic}

\usepackage[utf8]{inputenc}
\usepackage[top=1.1cm, bottom=1.1cm, left=2cm, right=2cm]{geometry}
\usepackage[french]{babel}
\usepackage[style=ieee-alphabetic, giveninits=true, maxnames=4, minnames=4]{biblatex}
\addbibresource{cv_abrv.bib}

\DeclareNameFormat{author}{%
  \ifthenelse{\equal{\namepartfamily}{Naveau}}
    {\textbf{\namepartgivenabbr\addspace\namepartfamily}}
    {\namepartgivenabbr\addspace\namepartfamily}%
  \ifthenelse{\value{listcount}<\value{liststop}}
    {\addcomma\addspace}
    {}%
}

\usepackage{etoolbox}
\newtoggle{myrefs}
\newcommand{\myref}[1]{\iftoggle{myrefs}{\textbf{#1}}{#1}}

\name{Maximilien}{Naveau}
\title{Ingénieur de recherche en robotique humanoïde}
\address{Toulouse}{}
\extrainfo{Permis B -- Français}
\photo[5cm]{../figures/photo_officiel_trunc.jpg}

\renewcommand*{\firstnamefont}{\fontsize{30pt}{20}\mdseries\upshape}
\renewcommand*{\familynamefont}{\fontsize{30pt}{20}\mdseries\upshape}
\newcommand{\items}{\item[*] \hspace{2mm}}
\newcommand{\todo}[1]{\textcolor{red}{#1}}

\begin{document}

\maketitle

\section{Expérience professionnelle}

\cventry{2021-2025}{Ingénieur de recherche}{PAL-FRANCE}{Toulouse/Caen}{France}
{
  \begin{itemize}
    \items Télétravail à Caen, France
    \items Coordinateur intégrateur du projet européen \textbf{AGIMUS}
    \items Développement logiciel pour des applications robotiques
  \end{itemize}
}

\cventry{2020-2021}{Ingénieur de recherche}{LAAS-CNRS}{Toulouse}{France}
{
  \begin{itemize}
    \items Coordinateur scientifique du projet européen \textbf{Memmo}
  \end{itemize}
}

\cventry{2016-2020}{Post-Doc}{MPI-IS}{Tübingen}{Allemagne}
{
  \begin{itemize}
    \items Implémentation d'une architecture de contrôle basée sur le dynamic-graph (développé au LAAS-CNRS). L'architecture couvre les logiciels temps réel (bas niveau) jusqu'à l'intégration ROS.
    \items Développement d'algorithmes de locomotion pour robots humanoïdes et quadrupèdes
    \items Intégration des logiciels sur le robot humanoïde \emph{Athena} et sur le robot quadrupède \emph{Solo}
    \items Mise en place d'une intégration continue pour les codes centraux de l'architecture de contrôle du laboratoire
    \items Publication en open source de ces codes (https://github.com/machines-in-motion/)
  \end{itemize}
}

\cventry{2013-2016}{Doctorant}{LAAS-CNRS}{Toulouse}{France}
{
  \begin{itemize}
    \items Développement d'algorithmes de locomotion pour robots humanoïdes
    \items Recherche en collaboration avec des partenaires étrangers sur des thématiques interdisciplinaires autour du mouvement des systèmes anthropomorphes et du mouvement humain, dans le cadre du projet européen Koroibot
    \items Intégration des logiciels sur le robot humanoïde \emph{HRP-2}
  \end{itemize}
}

\cventry{Mars 2013 à Septembre 2013}{Stagiaire}{CEA Saclay}{Gif-sur-Yvette}{France}
{
  \begin{itemize}
    \items Développement d'un algorithme d'évitement d'obstacles par stéréovision
    \items Intégration du système AVISO sur le robot ASSIST
  \end{itemize}
}

\cventry{Octobre 2012 à Février 2013}{Élève}{Supméca Toulon}{}{France}
{
  \begin{itemize}
    \items Conception, fabrication et instrumentation d'un navire à voile autonome, travail d'équipe (19 personnes) sur 5 mois
  \end{itemize}
}

\cventry{Septembre 2012 à Janvier 2013}{Stagiaire}{Université de Birmingham}{}{Royaume-Uni}
{
  \begin{itemize}
    \items Développement d'un algorithme de compliance active d'un bras manipulateur Kuka, travail en autonomie sur 5 mois, dans le cadre du projet GeRT (Generalizing Robot Manipulation Tasks)
  \end{itemize}
}

\section{Formations}
\cventry{2013 -- 2016}{Doctorat en robotique}
{LAAS-CNRS, à Toulouse (31)}
{Université Paul-Sabatier, Toulouse III}
{Encadré par Olivier Stasse}
{Stratégies de marche avancées, et inspirées de l'être humain, pour les robots humanoïdes}
\cventry{2013}{Master en vision et commande}{Université de Toulon (83)}{}{}{}
\cventry{2010 -- 2013}{Diplôme d'ingénieur}
{Supméca Toulon (83)}{}{}
{Spécialité : robotique et systèmes mécatroniques}
\cventry{2008 -- 2010}{Classe préparatoire aux grandes écoles}{Lycée Victor Hugo}{à Caen (14)}{}{Spécialité : Physique Sciences de l'Ingénieur}
\cventry{2008}{Baccalauréat S}{Lycée Allende}{à Hérouville St-Clair (14)}{}{Spécialité : SVT et Mathématiques}
\cventry{Langue}{Anglais}{lu, écrit, parlé}{TOEIC 865/990 2012}{}{
  Publications dans des revues internationales, stage ingénieur de 6 mois à Birmingham, Royaume-Uni
}
\cventry{}{Autres langues}{notions de chinois, italien, allemand}{}{}{}
\cventry{Informatique}{Logiciels}{cmake, git, eigen, numpy, ROS, Bamboo, LAAS-CNRS/Gepetto Team Framework}{}{}{}
\cventry{}{Langages}{C/C++, python, bash, nix}{}{}{}

\section{Publications scientifiques}

Tous les articles sont disponibles via l'archive ouverte HAL :
\url{https://hal.archives-ouvertes.fr/}

\large{\underline{Thèse de doctorat :}}
\begin{itemize}
\items \verysmall{\fullcite{phd:mnaveau:2016}}
\end{itemize}

\large{\underline{Articles de revues :}}
\begin{itemize}
\items \verysmall{\fullcite{mukovskiy:tro:2016}}
\items \verysmall{\fullcite{orthey:homotopic:2015}}
\items \verysmall{\fullcite{naveau:ral:2016}}
\items \verysmall{\fullcite{clever:ral:2017}}
\end{itemize}

\large{\underline{Articles de conférences :}}
\begin{itemize}
\items \verysmall{\fullcite{ramirez:ichr:2016}}
\items \verysmall{\fullcite{matan:biorob:2016}}
\items \verysmall{\fullcite{carpentier:icra:2016}}
\items \verysmall{\fullcite{kudruss:ichr:2015}}
\items \verysmall{\fullcite{naveau:ichr:2014}}
\items \verysmall{\fullcite{stasse:ichr:2014}}
\end{itemize}

\large{\underline{Revues d'articles scientifiques :}}
\begin{itemize}
\items IJRR, IEEE T-RO, IEEE RA-L, ICRA, IROS, Humanoids
\end{itemize}

\vspace*{1cm}

Tous les articles publiés peuvent être téléchargés depuis ce lien :
\url{http://projects.laas.fr/gepetto/index.php/Publications/ByAuthor?author=Maximilien_Naveau}

\section{Centres d'intérêt}

\cventry{Musique}{Instruments}{Batterie (10 ans), Basse (1 an)}{}{}{}
\cventry{}{Création de 2 groupes}{"MARACLAAS" et "Open Doors"}{Concerts dans différents lieux en 2012-2013 à Toulon (France) et festival de musique en juin 2015 à Toulouse (France)}{}{}

\end{document}