% Exemple de CV utilisant la classe moderncv
% Style classic en bleu
% Article complet : http://blog.madrzejewski.com/creer-cv-elegant-latex-moderncv/

% mail a robert.martinez@progress-rh.com
% mettre numero de telephone

\documentclass[11pt,a4paper]{moderncv}
\moderncvtheme[blue]{classic}                
                
\usepackage[utf8]{inputenc}
%\usepackage{geometry}
\usepackage[top=1.1cm, bottom=1.1cm, left=2cm, right=2cm]{geometry}
\usepackage[english]{babel}

%\usepackage{bibentry}
%\bibliographystyle{ieeetr}
\usepackage[style=ieee-alphabetic,giveninits=true]{biblatex}
\addbibresource{cv_abrv_anonymous.bib}

% Modern biblatex way to bold a specific author in bibliography
\DeclareNameFormat{author}{%
  \ifboolexpr{ test {\ifcurrentname{author}} and test {\ifnumequal{\value{listcount}}{1}} }%
    {\ifstrequal{\namepartfamily}{Fa}{\textbf{Fa}, }{\namepartfamily, }}%
    {\namepartfamily, }%
  \namepartgiven
  \ifthenelse{\value{listcount}<\value{liststop}}{\addcomma\space}{}%
}

\DefineBibliographyExtras{french}{\restorecommand\mkbibnamelast}
\newcommand*{\mkbibnamelast}[1]{#1}

% Largeur de la colonne pour les dates
\setlength{\hintscolumnwidth}{2.5cm}

\firstname{Mushu}
\familyname{Fa}
\title{PhD student in humanoid robotics}              
\address{chez Mulan Fa, Chine}{}    
\email{mushu.fa@mulan.com}
%\homepage{www.google.com}
\mobile{00 00 00 00 00}
\extrainfo{age : 900 -- Driving license -- Chinese citizen}
\photo[4cm]{../figures/mushu.png}

\renewcommand*{\firstnamefont}{\fontsize{30pt}{20}\mdseries\upshape}
\renewcommand*{\familynamefont}{\fontsize{30pt}{20}\mdseries\upshape}
\newcommand{\items}{\item[*] \hspace{2mm}}
\newcommand{\todo}[1]{\textcolor{red}{#1}}

\begin{document}

%\vspace*{1cm}

\maketitle

\vspace*{1cm}

\section{Work Experience}
\cventry{2021-2025}{Research Engineer}{PAL-FRANCE}{Toulouse/Caen}{France}
{
  \begin{itemize}%
    \items Remote-working in Caen, France
    \items Integrator coordinator of the European project \textbf{AGIMUS}.
    \items Software development for robotic applications.
  \end{itemize}
}
\cventry{2020-2021}{Research Engineer}{LAAS-CNRS}{Toulouse}{France}
{
  \begin{itemize}%
    \items Scientific coordinator of the European project \textbf{Memmo}.
  \end{itemize}
}
\cventry{2016-2020}{Post-Doc}{MPI-IS}{T\"ubingen}{Allemagne}
{
  \begin{itemize}%
    \items impl\'ementation d'une architecture de contr\^ole bas\'ee sur le
    dynamic-graph (d\'evelopp\'e au LAAS-CNRS). L'architecture comprend
    les logiciels temps r\'eel (bas niveau) jusqu'\`a l'int\'egration ROS.
    \items d\'eveloppement d'algorithmes de locomotion pour robots humano\"ides
    et quadrup\`edes.
    \items int\'egration des logiciels sur le robot humano\"ide \emph{Athena} et
    sur le robot quadrup\`ede \emph{Solo}.
    \items mise en place d'une int\'egration continue, des codes qui sont au
    coeur de l'architecture de contr\^ole du laboratoire.
    \items mise en open source de ces m\^eme codes
    (https://github.com/machines-in-motion/).
  \end{itemize}
}
\cventry{2013-2016}{PhD in Robotics}{LAAS-CNRS}{Toulouse}{France}
{
\begin{itemize}%
\items Research in collaboration with partners on inter-disciplinary topics in the frame of the\\ \textbf{European~project~KoroiBot}.
\items Use of \textbf{linear} and \textbf{nonlinear optimization} for motion generation of humanoid robots.
\items Application of \textbf{human inspired} optimization costs and human inspired trajectory generation on human robots.
\items Integration of \textbf{embedded real time} applications on the \textbf{HRP-2} humanoid robot.
\items Design and implementation of novel algorithms for \textbf{walking} and \textbf{multicontact locomotion} of humanoid robot
\items Main developer of jrl-walkgen\\ \url{https://github.com/jrl-umi3218/jrl-walkgen.git} (branch devel)
\newline{}
\end{itemize}
}
%
\cventry{Mars 2013\\until September 2013}{Training}{CEA Saclay}{Gif-sur-Yvette}{France}
{
\begin{itemize}%
\items Obstacle avoidance of a robotic arm using stereo-vision,
\items integration of the AVISO system on the robot ASSIST.
\newline{}
\end{itemize}
}
%
%
\cventry{October 2012\\until February 2013}{Student project}{Supm\'eca Toulon}{}{France}
{
\begin{itemize}%
\items Mechanical design, fabrication and instrumentation of an autonomous sail boat,
\items team work (19 students) for 5 months,
\newline{}
\end{itemize}
}
%
%
%
\cventry{September 2012\\until January 2013}{Training}{University of Birmingham}{}{United-Kingdom}
{
\begin{itemize}%
\items Implementation of an admittance control on a 6 degree of freedom Kuka robotic arm,
\items 5 months of autonomous work, in the frame of the GeRT (Generalizing Robot Manipulation Tasks) project
\newline{}
\end{itemize}
}

\newpage

\section{Education}
\cventry{2013 -- 2016}{PhD in Robotics}
{in LAAS-CNRS}
{Universit\'e Paul-Sabatier, Toulouse III}
{supervised by Chi-Fu. \textit{"Advanced human inspired walking strategies for humanoid robots".} Defense the 28$^{th}$ of September 2016}
{}
\cventry{2010 -- 2013}{2 master degrees}
{Engineer Diploma (Supm\'eca Toulon) AND \emph{Master in vision and command} (University of Toulon)}
{France}
{}
{Specialty : robotics and mechatronic systems}
%
\cventry{2008 -- 2010}{Preparatory classes to enter the French Engineering Schools + Baccalaur\'eat (A-level)}{in Caen, France}{}{}{Specialty : Mathematics, Physics, Engineer science}%
%
\cventry{Language}{English}{fluent : read, written, spoken}{TOEIC 865/990 2012}{}{Publications in international journals, training for 6 months in Birmingham UK}
\cventry{}{Other languages}{notion of Chinese and Italian}{}{}{}
%
\cventry{Informatics}{Software}{cmake, git, OpenCV, Matlab(Simulink), LabView, CATIA(V5R20), Adams}
{}{}{}
%
\cventry{}{Language}{C/C++,python,bash}{}{}{}


\section{Scientific animation}

\cventry{Mai 2015}{Congress}{Doctoral School EDSYS}{}{France}
{
\begin{itemize}
\items Organization of the annual doctoral school congress.
\items Management of the logistic of the congress for the 2 days.
\end{itemize}
}


\section{Publications}

\large{\underline{PhD thesis :}}

\begin{itemize}%
\items \fullcite{phd:mnaveau:2016}
\end{itemize}

\large{\underline{Journal papers :}}

\begin{itemize}%
\items \fullcite{mukovskiy:tro:2016}
\items \fullcite{orthey:homotopic:2015}
\items \fullcite{naveau:ral:2016}
\items \fullcite{clever:ral:2017}
\end{itemize}%

%vspace*{0.3cm}

\large{\underline{Conference papers :}}

\begin{itemize}%
\items \fullcite{ramirez:ichr:2016}
\items \fullcite{matan:biorob:2016}
\items \fullcite{carpentier:icra:2016}
\items \fullcite{kudruss:ichr:2015}
\items \fullcite{naveau:ichr:2014}
\items \fullcite{stasse:ichr:2014}
\end{itemize}

%vspace*{0.3cm}
%\subsection{In submission}

\large{\underline{Reviewer in :}}
\begin{itemize}%
\items IJRR, IEEE T-RO, IEEE RA-L, ICRA, IROS, Humanoids.
\end{itemize}

\vspace*{1cm}

Published papers can all be downloaded from this link :
\url{http://projects.laas.fr/gepetto/index.php/Publications/ByAuthor?author=mushu_fa}\\

\section{Interests}

\cventry{Music}{Instruments}{Drum (10 years), Bass (1 year)}{}{}{}
\cventry{}{Creation of 2 bands}{"MARACLAAS" and "Open Doors" }{ concerts in different places in 2012-2013 in Toulon (France) + music festival in June 2015 in Toulouse (France)}{}{}

\end{document}
