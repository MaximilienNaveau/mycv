%%%%%%%%%%%%%%%%%%%%%%%%%%%%%%%%%%%%%%%%%
% Long Lined Cover Letter
% LaTeX Template
% Version 1.0 (1/6/13)
%
% This template has been downloaded from:
% http://www.LaTeXTemplates.com
%
% Original author:
% Matthew J. Miller
% http://www.matthewjmiller.net/howtos/customized-cover-letter-scripts/
%
% License:
% CC BY-NC-SA 3.0 (http://creativecommons.org/licenses/by-nc-sa/3.0/)
%
%%%%%%%%%%%%%%%%%%%%%%%%%%%%%%%%%%%%%%%%%

%----------------------------------------------------------------------------------------
%	PACKAGES AND OTHER DOCUMENT CONFIGURATIONS
%----------------------------------------------------------------------------------------
\documentclass[10pt,stdletter,dateno,sigleft]{newlfm} % Extra options: 'sigleft' for a left-aligned signature, 'stdletternofrom' to remove the from address, 'letterpaper' for US letter paper - consult the newlfm class manual for more options

\usepackage{charter} % Use the Charter font for the document text

\newsavebox{\Luiuc}\sbox{\Luiuc}{\parbox[b]{1.75in}{\vspace{0.5in}
\includegraphics[width=1.2\linewidth]{logo.png}
}} % Company/institution logo at the top left of the page
\makeletterhead{Uiuc}{\Lheader{\usebox{\Luiuc}}}

\newlfmP{sigsize=30pt} % Slightly decrease the height of the signature field
\newlfmP{addrfromphone} % Print a phone number under the sender's address
\newlfmP{addrfromemail} % Print an email address under the sender's address
\PhrPhone{Phone} % Customize the "Telephone" text
\PhrEmail{Email} % Customize the "E-mail" text

%\lthUiuc % Print the company/institution logo

%----------------------------------------------------------------------------------------
%	YOUR NAME AND CONTACT INFORMATION
%----------------------------------------------------------------------------------------

\namefrom{Maximilien Naveau} % Name

\addrfrom{
\today\\[12pt] % Date
6 rue Ren\'e Duguay Trouin\\ % Address
Toulouse, France 31400
}

\phonefrom{(+33) 06 48 36 37 42} % Phone number

\emailfrom{maximilien.naveau@gmail.com} % Email address

%----------------------------------------------------------------------------------------
%	ADDRESSEE AND GREETING/CLOSING
%----------------------------------------------------------------------------------------

\greetto{Dear Mr. Ludovic Righetti,} % Greeting text
\closeline{Sincerely yours,} % Closing text

\nameto{Mr. Ludovic Righetti} % Addressee of the letter above the to address

\addrto{
Max-Planck-Institute for Intelligent Systems\\
Autonomous Motion Department\\
Paul-Ehrlich-Str. 15\\
72076 T�bingen, Germany
}

%----------------------------------------------------------------------------------------


%----------------------------------------------------------------------------------------
%	HEADER FOR USING BIBLIOGRAPHY
%----------------------------------------------------------------------------------------

\usepackage[style=ieee-alphabetic,firstinits=true]{biblatex}
\bibliography{cv}
\usepackage{hyperref}

%----------------------------------------------------------------------------------------


\begin{document}
\begin{newlfm}

%----------------------------------------------------------------------------------------
%	LETTER CONTENT
%----------------------------------------------------------------------------------------

I am currently a french PhD student in LAAS-CNRS working on advanced walking strategies for humanoid robots.
My defense will take place the 28$^{th}$ of September and I am looking for a Post-Doc position for about two years.
I am willing to apply to your group as a french Post-Doc in Humanoid Walking Control.

It would be a fantastic experience to work with you on generalized locomotion for humanoid robots.
A core topic of my PhD thesis was to use linear and nonlinear optimization methods for generating walking motion trajectories.
In addition to that I was in charge of deploying those algorithms on the HRP-2 humanoid robot.
In the framework of the European Project KoroiBot I had fruitful collaborations with the University of Heidelberg.
We worked on the python and C++ implementation of a real time walking pattern generator able to handle nonlinear constraints.
It optimizes simultaneously the position and orientation of the foot and an extension of it would be to use mixed integer solvers as in \textbf{[1]}.
We also worked on generalized locomotion with multiple non coplanar contacts.
With specialists in human motion from the Weizmann Institute of Science and the University of Tuebingen we used properties extracted from human motions to generate robust and versatile locomotion.
Furthermore I really would love to work on the Valkyrie humanoid robot.

I am available at your convenience for any further information.

My under review manuscript : \url{https://cloud.laas.fr/index.php/s/ioSMVXlONWJg3v2}\\
My personal web page : \url{http://projects.laas.fr/gepetto/index.php/Members/MaximilienNaveau}\\
\textbf{[1]} \fullcite{deits:ichr:14}\\

%----------------------------------------------------------------------------------------

\end{newlfm}

\end{document}
