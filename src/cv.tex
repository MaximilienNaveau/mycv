% Exemple de CV utilisant la classe moderncv
% Style classic en bleu
% Article complet : http://blog.madrzejewski.com/creer-cv-elegant-latex-moderncv/

% mail a robert.martinez@progress-rh.com
% mettre numero de telephone

\documentclass[11pt,a4paper]{moderncv}
\moderncvtheme[blue]{classic}                
                
\usepackage[utf8]{inputenc}
%\usepackage{geometry}
\usepackage[top=1.1cm, bottom=1.1cm, left=2cm, right=2cm]{geometry}
\usepackage[english]{babel}

%\usepackage{bibentry}
%\bibliographystyle{ieeetr}
\usepackage[style=ieee-alphabetic,firstinits=true]{biblatex}
\bibliography{cv}

\DefineBibliographyExtras{french}{\restorecommand\mkbibnamelast}
\newcommand*{\mkbibnamelast}[1]{#1}

% Largeur de la colonne pour les dates
\setlength{\hintscolumnwidth}{2.5cm}

\firstname{Maximilien}
\familyname{Naveau}
\title{Doctorant en robotique humano\"ide}              
\address{6 rue Ren\'e Duguay Trouin 31400 Toulouse}{}    
\email{maximilien.naveau@gmail.com}
%\homepage{www.google.com}
\mobile{06 48 36 37 42} 
\extrainfo{25 ans -- Permis B}
\photo[3cm]{../figures/DSC00009.jpeg}

\renewcommand*{\firstnamefont}{\fontsize{30pt}{20}\mdseries\upshape}
\renewcommand*{\familynamefont}{\fontsize{30pt}{20}\mdseries\upshape}
\newcommand{\items}{\item \hspace{2mm}}
\newcommand{\todo}[1]{\textcolor{red}{#1}}

\begin{document}

\vspace*{1cm}

\maketitle

\vspace*{1cm}

\section{Formations}
\cventry{2013 -- 2016 (Soutenenace \`a venir)}{Doctorat en robotique}
{au LAAS-CNRS}
{Universit\'e Paul-Sabatier, Toulouse III}
{encadr\'e par Olivier Stasse.}
{"Strat\'egies de marche avanc\'ees, et inspir\'ees
de l'\^etre humain, pour les robots humano\"ides".}
\cventry{2008 -- 2013}{Double dipl\^ome \'equivalent M2}
{Dipl\^ome d'\emph{ing\'enieur} (Supm\'eca Toulon) et de \emph{Master en vision et commande} (Universit\'e de Toulon}
{Campus de Toulon (83)}
{}
{Sp\'ecialit\'e : robotique et syst\^eme m\'ecatronique}
%
\cventry{2005 -- 2010}{Classe pr\'eparatoire au grandes \'ecoles + Baccalaur\'eat}{\`a Caen (14)}{}{}{Sp\'ecialit\'e : Physique Science de l'Ing\'enieur}%
%
\cventry{Langue}{Anglais}{lu, \'ecrit, parl\'e}{TOEIC 865/990 2012}{}{Publications dans des revues internationales, stage ing\'enieur de 6 mois \`a  Birmingham, Royaume Unis}
\cventry{}{Autres langues}{notion de chinois, italien}{}{}{}
%
\cventry{Informatique}{Logiciels}{cmake, git, OpenCV, Matlab(Simulink), LabView, CATIA(V5R20), Adams}
{}{}{}
%
\cventry{}{Language}{C/C++,python,bash}{}{}{}

\section{Exp\'erience professionnelle}
\cventry{2013-2016}{Doctorant}{LAAS-CNRS}{Toulouse}{France}
{
\begin{itemize}%
\items d\'eveloppement d'algorithmes de marche de robot humano\"ides et locomotion multi-contact,
\items recherche en collaboration avec des partenaires \'etrangers sur des th\'ematiques interdisciplinaires autour du mouvement des syst\`emes anthropomorphes et du mouvement humain, dans le cadre du projet europ\'een Koroibot.
\items int\'egration logiciels sur le robot humano\"ide \emph{HRP-2}.
\newline{}
\end{itemize}
}
%
%
\cventry{Mars 2013\\\`a Septembre 2013}{Stagiaire}{CEA Saclay}{Gif-sur-Yvette}{France}
{
\begin{itemize}%
\items d\'eveloppement d'un algorithme d'\'evitement d'obstacles par st\'er\'eovision,
\items int\'egration du syst\`eme AVISO sur le robot ASSIST.
\newline{}
\end{itemize}
}
%
%
\cventry{Octobre 2012\\\`a F\'evrier 2013}{\'El\`eve}{Supmeca Toulon}{}{France}
{
\begin{itemize}%
\items Conception, fabrication et instrumentation d'un navire \`a voile autonome,
\items travail d'\'equipe (19 personnes) de 5 mois,
\newline{}
\end{itemize}
}
%
%
%
\cventry{Septembre 2012\\\`a Janvier 2013}{Stagiaire}{Universit\'e de Birmingham}{}{Royaume-Uni}
{
\begin{itemize}%
\items D\'eveloppement d'un algorithme de compliance active d'un bras manipulateur Kuka,
\items travail en autonomie de 5 mois, dans le cadre du projet GeRT (Generalizing Robot Manipulation Tasks)
\newline{}
\end{itemize}
}

\newpage


\section{Publications Scientifiques}

Tout les papiers peuvent \^etre t\'el\'echarg\'es \`a cette adresse :
\url{http://projects.laas.fr/gepetto/index.php/Publications/ByAuthor?author=Maximilien_Naveau}\\


\subsection{Revues Scientifiques}

\begin{itemize}%
\items \fullcite{naveau_ral_2016}
\end{itemize}%

\vspace*{0.3cm}
\subsection{Articles de Conf\'erences}

\begin{itemize}%
\items \fullcite{matan_biorob_2016}
\items \fullcite{carpentier_icra_2016}
\items \fullcite{kudruss_ichr_2015}
\items \fullcite{naveau_ichr_2014}
\items \fullcite{stasse_ichr_2014}
\end{itemize}

\vspace*{0.3cm}
\subsection{Articles en Cours de Soumission}

\begin{itemize}%
\items \fullcite{mukovskiy_tro_2016}
\end{itemize}

\section{T\^aches Administratives}

\cventry{Mai 2015}{Doctorant}{\'Ecole Doctoral EDSYS}{}{France}
{
\begin{itemize}
\items Organisation du congr\`es de l'\'ecole doctorale EDSYS.
\items Gestion de la logistique pour les deux jours du congr\`es.
\end{itemize}
}


\section{Centres d'inter\^et}

\cventry{Musique}{Instruments}{Batterie (10 ans), Basse (1 an)}{}{}{}
\cventry{}{Cr\'eation de 2 groupes de musiques}{"MARACLAAS" et "Open Doors" }{ concerts dans diverses sc\`enes \`a  Toulon + f\^ete de la musique 2015 \`a  Toulouse}{}{}


\end{document}
